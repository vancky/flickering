% !Mode:: "TeX:UTF-8"
% Author: Rickjin (ZhihuiJin@gmail.com)
%
\chapter{Stirling 公式推导}
\label{chap:stirling}

\section{Poisson 分布的性质}

如果 $X \sim Poisson(\lambda)$ 分布, 则
$$ P(X = k) = \frac{\lambda^k}{k!}{e^{ - \lambda }} $$
$Poisson$ 分布具有如下两个性质
\begin{itemize}
\item $Poisson(\lambda)$ 分布的均值和方差都是 $\lambda$
\item $Poisson$ 分布具有可叠加性。如果$X,Y$为独立随机变量,且 $X\sim
Poisson(\lambda_1), Y\sim Poisson(\lambda_2)$, 则 $X+Y\sim Poisson(\lambda_1+\lambda_2)$。
\end{itemize}

这两个性质不需要复杂的数学证明,只要知道二项分布 $B(n,p) \rightarrow
Poission(\lambda)$ ($n\rightarrow \infty$ 且 $np=\lambda$),就可以简单的推理出
来。

因为二项分布 $B(n,p)$ 的均值为 $np$, 方差是 $np(1-p)$, 当$n\rightarrow
\infty$ 且 $np=\lambda$ 时,显然有 $np(1-p)\rightarrow np=\lambda$。

第二个性质也可以基于二项分布推导出来。假设 
$B(n,p_1) \rightarrow Poission(\lambda_1)$, 
$B(n,p_2) \rightarrow Poission(\lambda_2)$。
$X$ 可以用如下二项分布逼近:把 $[0,1]$ 区间分解为 $[0, 1/n], [1/n, 2/n],
\cdots, [1-1/n, 1]$ 这$n$个区间,每个区间最多只能发生一次事件,于是每个区间就对
应于一个成功概率为 $p_1$的贝努利实验; 同样的把$Y$ 在这$n$个区间做分解, 则每个
区间对应到一个成功概率为 $p_2$的贝努利实验, 所以对应到 $X+Y$ 恰好是每个区间发生
一次事件的概率是 $p_1 + p_2$, 所以 $X+Y$ 对应于 $Poission(n(p_1+p_2))=Poission(\lambda_1+\lambda_2)$。


\section{Stirling 公式的统计推导}

利用 $Poisson$ 分布的这两个特性,再加上中心极限定理,我们可以有一种简洁的方法推出 Stirling 公式。

假设 $X_1, X_2,\ldots, X_n$ 为服从$Poisson(1)$ 的独立随机变量,令 $S_n=\sum_{i=1}^n
X_i$, 则由 $Poisson$ 分布的可叠加性,$S_n \sim Poisson(n)$, 且
\begin{equation}
P\{ {S_n} = n\}  = \frac{{{e^{ - n}}{n^n}}}{{n!}}
\end{equation}
由于$S_n$ 的均值和方差都是 $n$, 由中心极限定理有 
$$ \frac{S_n - n}{{\sqrt n }} \sim N(0,1) $$
其密度函数为
$$ \displaystyle f(x)=\frac{1}{\sqrt{2\pi}}e^{-\frac{x^2}{2}} $$
所以,我们有如下推导
\begin{eqnarray}
\begin{array}{lll}
 P({S_n} = n) & = & \displaystyle P\{ n - 1 < {S_n} \le n\}  \\ 
              & = & \displaystyle P\{  - \frac{1}{{\sqrt n }} < \frac{{{S_n} - n}}{{\sqrt n }} \le 0\}  \\ 
 & \approx & \displaystyle \int_{ - \frac{1}{{\sqrt n }}}^0 {\frac{1}{\sqrt{2\pi}}{e^{ - \frac{{{x^2}}}{2}}}dx}  \\ 
 & \approx & f(0) [0 - ( - \frac{1}{{\sqrt n }})] \\
 & = & \displaystyle {\frac{1}{\sqrt{2\pi}}}{e^{ - \frac{{{0^2}}}{2}}}[0 - ( - \frac{1}{{\sqrt n }})] \\

 & = & \displaystyle {(2\pi n)^{ - \frac{1}{2}}} \\ 
\end{array}
\end{eqnarray}
由以上两式得
$$ \frac{{{e^{ - n}}{n^n}}}{{n!}} \approx {(2\pi n)^{ - \frac{1}{2}}} $$
于是
$$ n! \approx {n^{n + 1/2}}{e^{ - n}}\sqrt {2\pi } $$
这就是 Stirling 公式。
